\chapter{Introduction}

The main objective of this project was to develop an Android ``City Rally'' application, by reusing as much as possible the concepts, methods and tools presented in the Android course of Principles of Software Development. We named our application ``City Hunter''. City Hunter tries to present a city in a playful and funny way. The user is able to select a city from a list of cities or simply take the closest city near him. Once the user selected a city he has to solve one or more mysteries. A mystery consists in guessing a word related to a question concerning the particular city. However, to increase the chances in solving the given mystery and to increase his score, the user has to visit several checkpoints across the city which are nice spots of the city or famous squares/places. At each of these checkpoints, the user has to complete a challenge by answering a specific question or do a certain action with his device. If the user succeeds every challenge, then he/she is granted a clue which helps him to get closer to solve the particular mystery. Each of the challenges is illustrated on a map as well as the route to take to visit a particular challenge. The user can try to guess the answer of the mystery as often as he wants. However, the potential score that he can get decreases for each try he guesses wrong, starting at a maximum score of 500 points and going down to a minimum score of 100 points. Similar for the challenges, except that each challenge has a maximum number of tries. For each wrong try, the score is decreased by 100 points. The maximum score of a challenge depends on the maximum number of tries that user has available for a particular challenge. For example, a challenge with 2 tries has a maximum score of 200 points, whereas a challenge with 3 tries is more difficult and has also therefore 300 points. Once the user has consumed all of his tries and he still didn't succeed the challenge, he loses the challenge and is not allowed to play the challenge anymore.